\chapter{State of the art of JavaCard security}\label{chp:state-of-the-art}
% TODO explain APDU command response and the status words
% FIXME unify smart card and smartcard
% SDK Versions

% JavaCard technology can run on JavaCard technology-enabled devices



\section{Introduction to JavaCards}
    
% JavaCard
% Converter
%     input class files
%     input export files
%     output CAPs
%         an optionallz EXP files
% installation tool
% components
% Applets and packages
%     applets are JavaCard technology programs
% CAP file = applets, packages or both
%     custom u1, u2, u4 types
%     converter creates CAPs
%     Compact/Extended format
%     custom format, tags values
%     JAR archive

% Mangling with export files - sergei volotikin

% AID
%     managed by ISO, in the application we us 100*
%     page 53. uniqueness of AIDs

% Components?
%     CAP=set of component
%     Export Component
%     Applet Component
%     optional and required

% private and public applets/packages
% JavaCard allows multiple applets to run on the device

% byte code instructions
%     subset, same for many of the Java classes
%     exceptions -> status words?

% cite: The Java Card RE consists of an implementation of the Java Card virtual machine along
% with the Java Card API classes

% non-runtime security
%     off-card, chain of trust

% Runtime security
%     applet firewall X shareable interface
%     contexts, RE context
%     RE uses special things, that are not part of API

% Installation/deletion
% Card session
    

% FIXME Java Card is an open platform allowing multiple applications to run along each other
% FIXME go through cards
We will start by introducing Java Card technology and smart cards. Java Card technology is based on the Java programming language and allows programs created in Java to be executed on secure devices such as smart cards. Similarly to Java, Java Card system is made up from multiple components such as Java Card Virtual Machine (JCVM), Java Card Runtime Environment (JCRE), Java Card Application Programming Interface (JCAPI) and additionally Java Card Convertor(JCC).  Smart cards are pocket sized devices with an embedded microchip. Smart card's chip contains components, such as CPU, RAM and ROM and non-volatile memory, known from other computing devices. However, the components inside a smart card chip are much more limited in their capabilities. Typical size of a user RAM is around 12K. The size of non-volatile memory (that is a memory, that allows persistant data storage) is higher at around 16K and ROM is the biggest with 32-48K. Also, the computing power of a CPU inside a smart card is nowhere near the computing power of CPUs used in computers.

Java Card technology is released in multiple versions, the newest one, at the time of writing this thesis, is the version 3.1. The Java Card platform specifications can be accessed at~\cite{jcspecs31download}. Since the version 3.0 Java Card platform was split into Classic and Connected editions. The Connected edition targets devices, that have restricted resources, but are more powerful, than conventional smart cards described above. If we refer in our work to a version of Java Card Platform higher than 3.0 we will mean the Classid Edition.

% In our work we focus only on the Classic edition for the version 3.0 and newer of Java Card technology.

Each Java Card technology version is accompanied with a corresponding Java Card Software Development Kit (SDK). SDK equips Java Card developers with the tools needed for the development of Java Card programs. In the context of Java Card platform the programs are called Java Card applets, or simply applets. As we will show later in the text the behaviour of an particular applet with respect to security can differ across the SDKs (and smart cards it is installed on).

% Classic and connected version.

We call smart cards, that enable Java Card technology JavaCards. Because JavaCards are not that powerful they cannot make use of all the features of the Java programming langugage. For that reason, only a subset of Java is used for developing JavaCard programs. Specifically, the JCVM supports only a subset of Java Virtual Machine. For example, JCVM does not support Security Manager class, threads or variable-length arguments. Furthermore, only some Java data types, like \texttt{boolean}, \texttt{byte}, \texttt{short}, \texttt{int} and Objects or arrays, are supported --- this implies, that several byte code instructions for handling more complicated types are not supported (the complete subset is described in~\cite{jcspecs31download}).


\section{JavaCard applet life-time}\label{sec:jc:lifecycle}


% FIXME add note about cref, https://docs.oracle.com/javame/dev-tools/jme-sdk-3.0-win/html-helpset/z400008e37797.html, reference implementation or Java Card Platform Simulator


Java Card applets are developed in the Java programming language. The applet source files can define one or more Java packages. A package is referred to as an applet package if it contains a subclass of \class{javacard.framework.Applet} (there can be more than one). It is called a library package otherwise.  Similarly to Java, the source code is compiled into class files using Java compiler. The class files produced can now be tested and debugged.

Before the applet is downloaded to the target device the class files need to be converted using the JCC to a CAP (converted applet) file. 

The CAP file format is the representation of Java Card application and can contain one or more applet and library packages. The CAP files consist of several different CAP components. Java Card Converter produces a ``CAP file'' (it uses the \texttt{.cap} file extension), but it is actually a Java Archive file (JAR) that comprises of the individual CAP components. Some of the CAP components are required and some are optional.

The second input to the JCC after the class files are export files (EXP). Export files contain the name and linking information about the packages, that are imported in the applet's source code.
Apart from creating CAP files JCC can also create export files for future conversions or Java Card Assembly (JCA) files. JCA is human-readable version of the applet's bytecode.

Before the CAP file is downloaded to the target device an off-card verifier can be utilized to verify whether the file conforms to the specifications for a given SDK version. To load a converted CAP file onto a smart card we need a Card Acceptance Device (CAD; smart card reader). 

Once a JavaCard is inserted into a reader we can use an installation tool (see~\ref{subsubsec:gpp}) to load the CAP file onto the device. The smart card itself has an installation program, which needs to follow the specifications in~\cite{jcspecs31download}.
% FIXME the surroungding paragraphes!!

There are three steps, that need to take place before the new applet can be used. The applet's CAP file needs to be loaded on the target device, than linked and finally the applets \mintinline{bash}{install(byte[], short, byte)} method is invoked (if the \texttt{install} method is not implemented then applet's objects cannot be created initialized~\cite{jcspecs31download}). After the installation JavaCard RE interacts with the applet mostly through the methods \texttt{select}, \texttt{process}, \texttt{deselect}, \texttt{uninstall}.

% After all three steps proceed successfuly the applet is ready and can be used. In case of successful installation of an applet packgage the applet is made 

Applets and packages are identified on card by their Application ID (AID) defined in ISO 7816-5~\cite{jcspecs31download}. AID value is concatenated from Registered Application Provider Identifier (RID) and Proprietary Application Identifier Extension (PIX)~\cite{globalplatform}. The communication with the applet is facilitated through Application Protocol Data Unit (APDU). The layout of general APDU command and response is in table~\ref{tab:apdu}.
\begin{table}
    % \begin{tabular}{lc}
        % \begin{minipage}{.5\linewidth}
        \hfill
        \parbox{.45\linewidth}{
        \centering
            \begin{tabular}{@{}lllllll@{}}
                \toprule
                   \cla & \ins & \pone & \ptwo & \lc & \data & \len \\
                \midrule
                   1 & 1 & 1& 1& 1 & 00--\lc& 0-3 \\
                \bottomrule
            \end{tabular}
            }
        % \end{minipage} & 
        % \begin{minipage}{.5\linewidth}
        \hfill
        \parbox{.45\linewidth}{
        \centering
        \begin{tabular}{@{}lll@{}}
            \toprule
                \data & \texttt{SW1} & \texttt{SW2}\footnotemark \\
            \midrule
                max \len & 1 & 1 \\
            \bottomrule
        \end{tabular}
        }
        % \end{minipage}
        \caption{The general structure of APDU command on the left and APDU response on right~\cite{jcspecs31download}. The field names are in the top row and their length in the bottom one}
        \label{tab:apdu}
    % \end{tabular}
\end{table}

\footnotetext{ \texttt{SW1SW2} are called status word. The values of status words are defined in the \texttt{javacard.framework.ISO7816} package.}

When CAD wants to interact with a particular applet it issues a special APDU containing the applets AID. JavaCard RE deselects (suspends) the currently selected applet by calling its \texttt{deselect} method and calls the \texttt{select} method of the applet that matches the AID. Further APDU payloads are then relayed to the applet through its \texttt{process} method. Finally, the \texttt{uninstall method is called when a APDU requesting deletion of the applet is received by JCRE.

    The \cla field from~\ref{tab:apdu} is used to differentiate higher level functionality of the applet. The \ins field is used to trigger particular operation in the applet. The \texttt{process} method usually implementes a \texttt{switch} statement, that dispatches the \ins instructions to the respective methods. We will see the \ins field be used for communicating with the applets in POCs.

    % \subsection{Development Cycle of a JavaCards}
    % JavaCard technology is based upon the Java technology. Among other, Java gained its popularity thanks to the ability to run on multiple platforms. Java achieves platform independence with a middle layer of a Java Virtual Machine. Java source code is compiled into intermediate byte code, which is then executed by the virtual machine on the target device. Smart cards can be deceiving, because they look similar on the outside, however, they are produced by different manufactures. Therefore, smart card applications face the similar problem of platform independece as regular computer applications do. JavaCard technology uses similar approach to overcome this issue and achieve cross-platformity. JavaCard applications are not run by the smart card operating system, but by the JavaCard Virtual Machine (JCVM) instead. Similarly to Java, JavaCard application are compiled into JavaCard bytecode, which is then executed by JCVM.

% FIXME
% Communicating with the JavaCard applet.


% FIXME As we will see in the chapter~\ref{chp:results}.

% FIXME distinguish applet package and library package

    % Before we present some of the research done in the field of attacks against JavaCards we will describe the platform itself.
    % \subsection{What are JavaCards}
        % Smart cards are pocket sized devices with an embedded microchip. JavaCard technology is 

        % a form of smart card devices, that runs JavaCard Virtual Machine on top of the smart card operating system. Current smart cards have similar layout (TODO better word) as other computing devices, but its resources are much more limited. A typical smart card has 


% FIXME Java Card technology allows multiple applications to be installed on the smart card at the same time. This introduces a risk of one application accessing (potentially secret) data of another application.



    \section{Types of attacks and attacker models}

    As we will see in the list of related work~\ref{sec:related-work} the previous research distinguishes three main attack categories --- physical attacks, logical attacks and combined attacks. Physical attacks target the device physical components, such as CPU and memory, by for example shooting a laser beam at them. In \cite{Prpič2010thesis} the power traces caused by the power consumption due to the execution of the byte code instructions are stored and analyzed. A database of such traces can then be used to partially reverse engineer the instruction in future interception of power traces of unknown code execution.
    Logical attacks like~\cite{hogenboom} target the software implementation of JavaCards and try to exploit bugs in the implementation of JCRE and JCVM (or even the card OS). Combined attacks combine physical and logical attacks. Physical disturbance might allow skipping an internal validation of malicious applet, which would otherwise be prevented from execution as in~\cite{barbusecond}.

    % FIXME move elsewhere!!
    % \subsection{The attacker model}

        Because we are interested in attacks on the JavaCard platform we need to discuss the attacker model first. In general, the attacker model for the various attacks on JavaCards can differ. For physical attacks the attacker usually to have the targetted JavaCard in a long-term possession and may or may not need to be able to install additional applications on the JavaCard and to communicate with the card. For logical attacks, the attacker often needs to be able to install additional applications on the target device and then communicate with it (this might not require physical access to the JavaCard as demonstrated in~\cite{se:gemalto:part}). For combined attacks (the combination of the two previous) the adversary will probably need both the long-term access to the card and ability to install applets on it.
        In the second chapter we present the JavaCard Vulnerability Scanner, this tool currently supports multiple logical attacks and for each one of them the attacker model is the same. The adversary needs to be in the possession of the targed JavaCard, have the ability to install additional applets and issue APDU commands to it afterwards.

        % Smart card platform allows for multiple different attacker models. The goal of the attacker can be to retrieve secret information, or alter information on the card or maybe to block the card from functioning. There are three main categories of attacks against smart cards.
    % \subsection{Physical attacks}
        % Those attacks can be in general carried out agains any type of smart cards, they are not JavaCard specific. In general a common attack vector against any device is to interact with the given device in a manner it was not intended (or even thought of) by its creators. Similarly, physical attacks against smart cards abuse a smart card physical interface. One of the simplest attacks against contact smart cards is to intercept the communication between the smart card and the reader. First thing is to extend the reader with a dummy smart card, that is wired to a new reader slot. All communication will now pass through the wires, that are easily accessible. Smart cards don't have their own power source, instead they get a power from the reader (from power supply pin in case of cards with contacts, for contact-less cards it is more difficult). By using another external chip we can hook to the I/O wire and to the power supply one. The second chip listens on the I/O line and once it intercepts a communication, imagine a Personal Identification Number (PIN) being transmitted, it can alter the voltage on the power supply wire, which can maliciously affect the cards behaviour. Such a basic attack is quite naïve and fragile in the sense, that e.g.\ skipping particular check (the PIN validation) on the card is very difficult. However, the attack is not resource intensive. The author himself tested the attack with no interesting results, as the card always detected the voltage change and stopped its operation for several minutes.

        % More elaborate attack is to use oscilloscope and eavesdrope on one of the chosen wires. In~\ref{Reverse engineering of Java Card applets} was shown, how the underlying bytecode instructions, that are executed can be inferred from the powertrace.

        % Currently, the physical attacks feature much more elaborate setups, where particular regions of the memory of the card can be targetted with e.g.\ lasers, which can lead to a change of values in the memory (those changes can be done on the level of individual bytes/bits). If the attacker then knows where to hit, he might be able to change for example \texttt{0x00} byte into \texttt{0xff}, which might lead to passing some if condition, that would otherwise fail.

        % Physical attacks yield promising results, but due to their complexity and resource requirements we cannot include them in the testing framework. For this reason those attacks won't be discussed further in this text.
        % TODO over si, ze to jde i na bit
        % TODO ref https://is.muni.cz/th/gkgiy/bc.pdf

    % \subsection{Logical attacks}
        
    % \subsection{Combined attacks}

    %     physical, logical, combined
    % % \subsection{Types of attacks}
        
    \section{JavaCard defensive mechanisms}

    Java Card technology is build to be used in secure elements such as smart cards. Therefore it is natural, that several defensive mechanisms are in place to attempt to stop the attacker from misusing the devices to for example obtain information, that should be kept secret.

The techniques we are about to described are software based. However, smart cards need also to be tamper resistant, because several techniques can be utilized to compromise its security as described in~\cite{kommerling}.
        % Smart cards in general are developed to be used in situation, where security is an important if not the most important requirement. They are used in banking industry, person identification (national ID cards, passports)  or for example as access tokens to buildings. Due to those reasons it is obvious, that mechanisms defending the secret information (such as private keys used in Public Key Infrastructures) stored on those cards need to defended well against various attack vectors.

% \section{Defensive Measures Against the Attacks}
    % \subsection{The Compiler}

% The Java compiler is used to compile the Java Card source files into the class files. 
    \subsection{The off-card bytecode verifier}

% FIXME cite SDK
The various Java Card SDKs are equiped with tools used for off-card verifications of the CAP, EXP and JCA files~\cite{jcoffcardverifier}. After the CAP file conversion the off-card verification can take place. The internal integrity of the CAP file and the associated export files is verified according to the JCVM specifications~\cite{jcspecs31download}.

    \subsection{The on-card bytecode verifier}

    If the off-card byte code verification is not enforced a malicious CAP file can be created and later loaded onto a card. It is then up to the card, whether it will accept or reject such CAP file. As we will see in chapter~\cite{chp:results} the behaviour of JavaCards differ. Similar, discrepanies in the implementation of on-card installer and JavaCard RE are observed in~\cite{lanettrojan}. The on-card bytecode verification is mandatory for JavaCard 3.0 Connected Edition and higher~\cite{barbusecond}. 



    \subsection{Applet Firewall}

    JavaCard technology allows multiple applications to be installed alongside on a single devices~\cite{jcspecs31download}. To prevent the different applets to accidentaly (or malicious) access data of other applications the JCRE introduces in~\cite{jcspecs31download} Applet firewall. Different applets on JavaCard are assigned different spaces called contexts. The firewall acts as a boundary between those contexts. Each CAP file of an applet package is assigned its own context, library packages share the context of the creating applet instance. One CAP file can define multiple applets --- those will share the same contexts and can therefore access objects of each other. Applets from one context are denied by the firewall access to applets from a different context.

    There is one special context assigned to the JavaCard RE. It possesses higher privileges and can access contexts of other applets. Sometimes the applets from different contexts do need to access objects, that reside in different context. Theses cases are covered by the following mechanisms --- JCRE Entry Points Objects, Sharing Arrays, JCRE privileges, Shareable Interfaces. An example of JCRE Entry Point Object is the APDU object, that implments the buffer used for I/O operations with the CAD.


% FIXME
% JCRE contexts, applets from the same package share the context

\section{The related work}\label{sec:related-work}
    
    % 2001
    In~\cite{oncardleroy} \citeauthor{oncardleroy} first reviews how the traditional off-card bytecode verification algorithm works point out its limitations. Afterwards a novel verification algorithm is presented, that requires much less RAM and can therefore be ran on-card, however, it cannot verify all type-correct applets. But this limitation is fixed with off-card transformation of the applet and therefore all type-correct applets can be verified on-card.

    % 2007
    \citeauthor{Mostowski07testingthe} analyze in \cite{Mostowski07testingthe}, whether the implementation of JCRE Firewall Mechanism on real JavaCards complies with the JCRE specification version 2.2.2. Among other, they test accesses to foreign objects through shareable interfaces, calls to privileged API methods and context stack corruption through infinite recursive method call. They have developed the tool Java Card Firewall Tester, however, it is no longer available for download from~\cite{firewalltester}. They conclude, that all except one (regarding the shareable interface) violation of the JCRE specification are minor.

% 2009
\citeauthor{hogenboom} present an attack, that exploits the JavaCard type system. They use faulty implementation of transaction mechanism. This mechanism does not restore local array reference allocated within the trasnsaction to \texttt{null} when the transaction is aborted. An array \texttt{arraySlocal} of shorts can be used in the previous step and so after the abortion a \texttt{short[]} pointer is dangling in memory. Immediately after the transaction abortion a \texttt{byte[]} array \texttt{arrayB} is created in persistent memory, reclaiming the dangling pointer, which makes it valid for the current instance. However, the memory location pointed to by this pointer can be read as an array of shorts (throught the original \texttt{arraySlocal}). Which returns twice the size of the correctly allocated memory of \texttt{arrayB} (shorts are twice as big as bytes). A Proof of Concept (POC) code was presented as a part of the original paper~\cite{hogenboom}. It defines three instructions \prepareone, \preparetwo and \insreadmem. The first two instructions set up the necessary array references by exploiting the implementation bug explained above and the \insreadmem instructions attempts to read out the memory address at \texttt{P1P2} (sent as part of the APDU).

% 2010
\citeauthor{lanettrojan} present a working example of dumping a bigger parts JavaCard's memory in~\cite{lanettrojan}. A custom applet was developed and its CAP file's Method and Reference Location Components were modified before applet's installation. With such an applet the authors have been able to perform search and replace operations on memory regions not owned by the applet and so effectively broke the firewall mechanism. The attack was successfuly carried out on several real JavaCards.

    % analyzed the JCRE Firewall Mechanism with respect to the specifications in~\ref{}
    % \subsection{Combined Attacks and Countermeasures, 2010}
    In~\cite{barbufirst} introduce~\citeauthor{barbufirst} as a first of its kind a combined attack against the JavaCard 3.0 Connected Edition. They combine a fault injection (shooting a laser beam on the smart card chip) and forging object references using the \texttt{getfield} and \texttt{putfield} byte code instructions. The targetted Java Card technology requires on-card byte coder verifier, however, the presented attack shows, that this one-time (during installation) security check can be circumvented.
    % putfield, getfield


    % \subsection{Java Card Operand Stack: Fault Attacks, Combined Attacks, and Countermeasures, 2011}
    Combined attacks were further studied by~\citeauthor{barbusecond} in~\cite{barbusecond}. Firstly, a fault injection attack, that targets \texttt{if} statement branching is presented. Experimental results show success in around 70-80\% of the attempts to skip the expected branch and execute a different one. Then a way of breaking type safety through fault injection in the operand stack was proposed. Finally, three specific countermeasures against the presented attacks are introduced and compared.

    % \subsection{Performance Evaluation of Java Card Bytecodes, 2007/8}
    % \subsection{Java Card Virtual Machine Compromising from a Bytecode Verified Applet, 2015}

    % \subsection{Memory Forensics of a Java Card Dump, 2015}

    % \subsection{Logical attacks on secured containers of Java Card platform, 2017}

% \section{State of the art attacks}
    % \subsection{Good, Bad and Ugly Design of Java Card Security, 2016}
    % FIXME MAJOR TUDOO
    In his master's thesis~\citeauthor{sergei} presents several attacks against JavaCards.~\cite{sergei} describes maltitude of attacks. The most interesting ones are attacks on the secure containers such as \texttt{OwnerPIN}, the author first finds the object in memory, which is stored encrypted, then usses chosen plaintext cryptanalysis to recover, that the PIN is encrypted using ECB mode. The author further noticies similarities in the handling of \texttt{OwnerPIN} and \texttt{DESKey} objects and finds out, that the \texttt{OwnerPIN} can by decrypted easily. First, encrypted bytes of \texttt{OwnerPIN} are copied into a \texttt{DESKey} controlled by the attacker, then \texttt{getKey} method is called on the unsuspecting \texttt{DESKey} and the \texttt{PIN} is retrieved to the attacker.

    \section{Security Explorations}\label{subsec:security-explorations}

    In 2019 a company from Poland called Security Explorations\footnotemark released five notices~\cite{se:oracle:part1, se:oracle:part2, se:oracle:part3, se:gemalto:part1,se:gemalto:part2} that describe security vulnerabilities in JavaCards. Overall, they have reported 34 individual issues. The reports~\cite{se:oracle:part1, se:oracle:part2,se:oracle:part3} test the vulnerabilities on JavaCard Reference Implementation from Oracle~\cite{jcspecs31download} and cover issues like insufficient implementation of JCVM bytecode instructions and JCAPI methods or handling of CAP file and its insufficient verification. The individual issues have been verified by Security Explorations on the reference implementation (\texttt{cref}) of Java Card technology version 3.1 (currently available for download at~\cite{jcspecs31download}).

The second batch of reports~\cite{se:gemalto:part1, se:gemalto:part2} analyze Gemalto Java Card based products and identify vulnerabilities in the SIM Toolkit applet (STK applet). Those vulnerabilities are caused by unmanaged or leaking memory references. Most notable is the result in~\cite{se:gemalto:part2}, where the authors used the discovered vulnerability to load Java Card applet over-the-air to a target SIM card. Furthermore, they conclude that it should be possible to exploit the vulnerabilities from~\cite{se:gemalto:part1, se:gemalto:part2} in such a way, that a hidden backdoor is loaded onto a SIM card.


\footnotetext{The company has stopped operating, however, its lead security researcher Adam Gowdiak is now working under Adam Gowdiak Security Research (see \url{http://www.agsecurityresearch.com/}).}

    The vulnerability noticies~\cite{se:oracle:part1, se:oracle:part2, se:oracle:part3} are also accompanied by a POC~\cite{se:downloadpage}, that help to demonstrate the issues. The POCs comprise of Java source files, custom Gen Tool and several batch scripts.

    % The source files are used to produce different applets, which are further altered with the Java based Gen Tool (no manual edits of CAP files are needed). The accompanied batch scripts allow the user to see the vulnerabilites work against the Java Card Reference implementation. % ~\cite{}. FIXME

We will go through the POCs exploiting the vulnerabilites in~\cite{se:oracle:part1, se:oracle:part2, se:oracle:part3} in greater detail, because we have included them in \projectname. Each POC consists of two applets \appletscap and \vulnscaporig (from which a malicious \vulnscap is generated with the Gen Tool). Every POC then defines several instructions (explained later), that are used to exploit the particular vulnerability.

There aren't as many POCs as there are vulnerabilities discovered, because the POCs unite the vulnerabilities of similar nature. We will use the issue numbering as in the original reports. The following sections briefly describe the vulnerabilities from~\cite{se:oracle:part1}.

\subsection{POC \texttt{arraycopy}}\label{subsec:arraycopy-explanation}
Similarly, the JCAPI methods \arrayCopy and \arrayCopyNonAtomic from \mintinline[breaklines,breakafter=.]{python}{javacard.framework.Util} class~\cite{jcspecs31download} do not perform sufficient checks on input arguments. Both methods fail to prevent passing object instance (instead of an array) as an input argument. \cite{se:oracle:part1} mentions, that some similar methods do the input validation properly. The POC implements the instructions \readmem and \writemem, that can read and write memory through exploitation of one of the aforementioned methods.

\subsection{POC \texttt{arrayops}}
As was explained in~\ref{subsec:arraycopy-explanation}, it is not always the case, that array methods perform sufficient validation of the its arguments. A custom crafted object can be passed as an input argument instead of an array reference. Such object is then interpreted as an array of a very large size. The POC \arrayops exploits the methods \arrayFill, \arrayFillNonAtomic, \setShort, \setInt. The POC applet defines two instructions \readmem and \writemem, that allow to read and write outside of the allowed memory regions.

\subsection{POC \texttt{baload_bastore}}
The byte code instructions \baload and \bastore load byte or boolean from array, respectively store byte or boolean from array~\cite{jcspecs31download}. The array argument to \baload instruction is checked and Security Exception is raised if the array is an array of shorts, integers or objects. Otherwise, the execution proceeds. However, instead of an array the argument can be a single object and the checks pass as well. 
The POC code defines six different instructions --- \ping, \status, \setup are used to check and setup the vulnerability state. The instructions \readmem and \writemem demonstrate reading and writing memory. Final instruction \cleanup is used to clean the initial exploit setup. The~\cite{se:oracle:part1} state the impact of the vulnerability is ``compromise of memory safety / arbitrary read access to card memory''.

\subsection{POC \texttt{nativemethod}}
The \texttt{Method} CAP file component is used to describe methods, that are defined within the CAP file. The \mintinline[breaklines,breakafter]{python}{method_info.method_header_info} specify whether the method is native or not. The values in the field \texttt{method_header_info} can be changed and make this method a native one. This approach can be used to call native methods such as~\texttt{readByte}, \texttt{readShort}, \texttt{writeByte} and \texttt{writeShort}, that should be inaccessible to the current class.
The POC implements two instructions \nreadshort and \nwriteshort, that invoke the native methods \texttt{readShort} and \texttt{writeShort}, respectively. A short value can be read or written with those instructions, that would be otherwise inaccessible.


\subsection{POC \texttt{referencelocation}}
Some byte code instructions, like \getfield reference various fields and methods. Those references are stored as indeceses to the Constant Pool (\mintinline{python}{ConstantPool} CAP component), that contains tokens for the fields and methods at each index. The internal representation of those tokens is substituted for the indeces when a CAP file is loaded on JavaCard.~\cite{se:oracle:part1} refers to this process as ``linking''. The \texttt{ReferenceLocation} component stores the table of locations for each of the previously described substitutions.
Before, the ``linking'' is done, each token value is checked to assure, that its value does not go beyond the size for the object (recall, that the token values refer to object fields and methods). If the value overflows, it is trimmed. However, if an entry in the \texttt{ReferenceLocation} table is omitted, the corresponding check of the token value is skipped as well, thus allowing the \getfield instruction access beyond the object size.
The POC defines six instructions \getfieldins, \putfieldins\footnotemark, which allow to read and write the memory through a custom token value. 
\footnotetext{The \texttt{<T>} refers to the particular type of the underlying byte code instruction defined in~\cite{jcspecs31download}.}

\subsection{POC \texttt{staticfieldref}}
As explained in~\ref{sec:jc:lifecycle} CAP file comprises of different CAP components one of which is \mintinline{python}{StaticField} component. This component is referenced through \constantstaticfieldref item~\cite{jcspecs31download}. The \texttt{offset} value is not checked and can be changed, to point to arbitrary value. Then, through the use of byte instructions \getstatic\footnotemark and \putstatic the memory at the new \texttt{offset} can be read and written, respectively. The POC uses single instruction~\getstaticins, that returns the \texttt{offset} value, that was maliciously altered after the POC CAP file conversion.
\footnotetext{Both \getstatic and \putstatic represent several byte code instructions for particular types.}

\subsection{POC \texttt{swap_x}}
JCVM is stack based and uses frames to hold the information about the currently executed method, such as local variables and the operand stack. The byte code instruction \swapx allows to swap $M$ words with $N$ words, that reside in the operand stack directly below. The intention is to swap one or two words on the stack~\cite{jcspecs31download}. If larger values for $N$ are provided a potential return address or instruction override can happen.
The POC code has a single instruction \triggerswapx, that invokes the malformed \swapx instruction. According to~\cite{se:oracle:part1} the instruction is expected to return \mintinline{python}{0x1234}, however, when executed in JCRE for \texttt{cref} a crash is observed in~\cite{se:oracle:part1} instead. When we have tested this POC on real JavaCards we have received the expected response as is discussed in~\ref{subsec:swapx} on four of them.


\section{Related JavaCard testing tools}
Riscure, Dutch company located in Delft, has a product called JCworkBench, that can be used to perform a robust security of a real JavaCard as demonstrated in~\cite{jcworkbench, riscurejcworkbenchpdf}. However, it is not clear from Riscure's website~\cite{riscureweb}, whether it can still be purchase as it is not listed there anywhere.\footnotemark

\footnotetext{The tool was only found presumably for sale on a different website at \url{https://www.etesters.com/product/7E9C25F0-D8CC-0B99-67BC-8BF36D42FE30/smart-card-security-test-tool/}}

Aforementioned Java Card Firewall Tester briefly introduced in~\cite{Mostowski07testingthe} is no longer available for download at~\cite{firewalltester} and the download site does not seem to be maintained anymore.

Two notable JavaCard testing tools are JCAlgTest~\cite{jcalgtest} and\linebreak ECTester~\cite{ectester}, both developed at CRoCS~\cite{crocsweb}. However, both those tools assess the security and safety of the implementation of the cryptograhical primitives and algorithms and not the resistance of JCRE or JCVM to logical attacks.














% \section{Logical attacks described in detail}
%     % shareable interface
%     \subsection{The Attacker Model}
%     \subsection{Vulnerability Categories}
%     \subsection{APDU Buffer}
%     % \subsection{\texttt{static} fields}
%     \subsection{Illegal cast/type confusion}
%     SV 3.1.
%     Illegal casting of an arbitrary short value to a reference
%     Illegal casting of a class instance to an array

%     \subsection{Bytecode Instructions}
%     SE - Oracle
%     \texttt{baload}
%     \texttt{bastore}
%     \texttt{getfield_a}
%     \texttt{getfield_b}
%     \texttt{getfield_s}
%     \texttt{getfield_i}
%     \texttt{putfield_a}
%     \texttt{putfield_b}
%     \texttt{putfield_s}
%     \texttt{putfield_i}
%     \texttt{swap_x}

%     \subsection{JavaCard Virtual Machine API}
%     SE - Oracle
%     \texttt{javacard.framework.Util.arrayCopy}
%     \texttt{javacard.framework.Util.arrayCopyNonAtomic}
%     \texttt{javacard.framework.Util.arrayCompare}
%     \texttt{javacard.framework.Util.arrayFill}
%     \texttt{javacard.framework.Util.arrayFillNonAtomic}
%     \texttt{javacard.framework.Util.setShort}
%     \texttt{javacard.framework.Util.intx.JCint}

%     SV
%     3.3 Abuse of Transaction mechanism


%     \subsection{CAP files}
%     SE - Oracle
%     \texttt{CONSTANT_StaticFieldref_info}
%     \texttt{COMPONENT_ReferenceLocation}
%     \texttt{COMPONENT_Method}

%     \subsection{Subroutines}
%     SE - Oracle
%     \texttt{__checkMethod}
%     \texttt{_getLocalReference}
%     \texttt{_setLocalReference}
%     \texttt{_getLocalShort}
%     \texttt{_setLocalShort}
%     \texttt{_getLocalInt}
%     \texttt{_setLocalInt}


