\section{Introduction}
% \addcontentsline{toc}{chapter}{Introduction}

This thesis investigates the state of the art attacks against the JavaCard Virtual Machine Environment. The first chapter introduces the realm of JavaCards --- the platform itself and its security measures and mechanisms. Afterwards we go through a list of past research done in the field and finally discuss particular attacks from the presented work and literature. In the second chapter we introduce the testing framework called JavaCard Vulnerability Scanner that was developed as a main part of this thesis. We discuss its design, features and limitations. This tool uses some of the attacks presented in the first chapter and makes it simple to execute and evaluate them on a real physical JavaCard. Because this tool was developed with future attacks in mind we show a tutorial on how to extend the framework with new attacks. Finally, the third and last chapter puts JavaCard Vulnerability Scanner into practice and uses it to perform security analysis of several real JavaCards.

    % \subsection{List of abbreviations}
    %     \begin{enumerate}[align=left]
    \item[AID] Application ID
    \item[APDU] Application Protocol Data Unit
    \item[API] Application Programming Interface
    \item[Applet] Java Card applet
    \item[BCV] Byte Code Verifier
    \item[BSON] Binary JSON
    \item[CAD] Card Acceptance Device
    \item[CAP] Converted Applet
    \item[ECB] Electronic Cookbook
    \item[EMV] Europay-Mastercard-Visa
    \item[JCC] Java Card Converter
    \item[JCRE] Java Card Runtime Environment
    \item[JCSystem] Java Card System
    \item[JCVM] Java Card Virtual Machine
    \item[JSON] JavaScript Object Notation
    \item[NoSQL] "non-sql" or "no-relational" database
    \item[PIX] Proprietary Application Identifier Extension
    \item[POC] Proof of Concept
    \item[RID] Registered Application Provider Identifier
    \item[SDK] Software Development Kit
    \item[SQL] Structured Query Language
    \item[STK] SIM Toolkit applet
\end{enumerate}

    % \subsection{Typographic conventions}
    % Some parts of the text are typesetted differently for clarity, further more, colors are occasionally used to help with the readability. The typesetting rules are the following:

    %     \begin{enumerate}[align=left]
    %         \item[\mintinline{bash}{$}] as usual in UNIX-like operating systems (OS) the dollar sign denotes the shell prompt,
    %         \item[\texttt{output}] lines that are not starting with \mintinline{bash}{$}, but are typeset in the \texttt{monospace font} denote the output of previous command, if used inline the meaning is usually either a command, filepath or reference to a piece of source code (class name, function name, etc.),
    %         \item[\texttt{[\ldots]}] denotes that some or all of the (usually unnecessary) output was omitted for clarity,
    %         \item[\texttt{<value>}] the angle brackets denote a placeholder value that should be substituted for a particular value based on the context,
    %         \item[\texttt{\#}] a pound symbol marks lines, that contain further explanation or commentary
    %         \item[] some commands or filepaths like this \mintinline[breaklines,breakafter=/]{yaml}{one/are/too/long/and/need/to/span/across/multiple/lines}.
    %     \end{enumerate}
